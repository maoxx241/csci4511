\documentclass{article}

\usepackage{graphicx}

\newcommand{\BibTeX}{{\sc Bib}\TeX}

\begin{document}

\title{Writing Assignment 2}
\author{Qi Mao\\
  \texttt{maoxx241@umn.edu}}
\maketitle


This article is talking about an algorithm named Generalized Adaptive A*. 
The author wants to find an algorithm that can find the shortest paths in stat space where the action costs can change over time. 
The author design an algorithm based on A* and that can find the shortest paths in state spaces where the action costs can 
increase or decrease over time. It can deal with multiply situation such as start 
state changes, the goal state changes and the action cost changes. 
The strategy that the algorithm used is updated the heuristics lazily. 
The algorithm will only update the heuristics for the state that will be successor. 
This algorithm is also good at the problem which has a moving target. For moving target, 
it can find the shortest path with less expand node. The author sets experiments in the 
four-neighbor gird-worlds, one independent stationary-target or moving-target search 
problem per grid-world. On the moving-target search problems, 
Generalized Adaptive A* is the best algorithm. 
But on the stationary-target, this algorithm is not good at all situation. 

In my opinion, 
this algorithm is stronger enough with the moving-target problem. 
Because it not only used the shortest time, 
but also expanded the smallest node. 
But it is not stronger enough with the stationary-target problem.
So I think it's not a stronger enough algorithm, 
because it is not efficient in solving the problem of stationary-target.
It is just a variant of the a* algorithm, 
and in some cases it is not even as good as the a* algorithm.
For this experiment, 
It just used the random node for the start state and the goal state, 
and  choose the random nodes for block. 
For the design of the experiment, 
I think it can be randomly selected block some more, 
because for the size 300x300, 50 block may not cause a great impact.
So there is no dataset used for experiment. 
But we can write a program to verify the accuracy of the result of experiments. 

I think this paper is not easy to follow. 
The author defines lots of notations and variables. 
These two parts are not adjacent or on the same page. 
The notation is on the second part and the Variables is on the fourth part. 
It is difficult to follow when you need to check the definition of tree() 
and the definition of succ(s,a). For the pseudocode part, 
the most difficult part to understand is the 19 line of the ComputePath() function, 
it said “tree(succ(s,a)) := s”. To be honest, 
I don’t have an idea about what this line do and how can I implement that line. 
For another line, It is readable for me. 


The most interesting part for me is this algorithm can find the moving target with using the fewer nodes that expanded and runtime that cost.
This algorithm can be used for those looking for moving targets. 
That means it will be useful on the question such as map navigation.


123\cite{greenwade93}
123\cite{rahtz89}
123\cite{patashnik88}
123\cite{lamport94}
123\cite{M}


\bibliographystyle{plain}
\bibliography{writing2}

\end{document}